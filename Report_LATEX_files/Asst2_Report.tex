% This is "sig-alternate.tex" V2.1 April 2013
% This file should be compiled with V2.5 of "sig-alternate.cls" May 2012
%
% This example file demonstrates the use of the 'sig-alternate.cls'
% V2.5 LaTeX2e document class file. It is for those submitting
% articles to ACM Conference Proceedings WHO DO NOT WISH TO
% STRICTLY ADHERE TO THE SIGS (PUBS-BOARD-ENDORSED) STYLE.
% The 'sig-alternate.cls' file will produce a similar-looking,
% albeit, 'tighter' paper resulting in, invariably, fewer pages.
%
% ----------------------------------------------------------------------------------------------------------------
% This .tex file (and associated .cls V2.5) produces:
%       1) The Permission Statement
%       2) The Conference (location) Info information
%       3) The Copyright Line with ACM data
%       4) NO page numbers
%
% as against the acm_proc_article-sp.cls file which
% DOES NOT produce 1) thru' 3) above.
%
% Using 'sig-alternate.cls' you have control, however, from within
% the source .tex file, over both the CopyrightYear
% (defaulted to 200X) and the ACM Copyright Data
% (defaulted to X-XXXXX-XX-X/XX/XX).
% e.g.
% \CopyrightYear{2007} will cause 2007 to appear in the copyright line.
% \crdata{0-12345-67-8/90/12} will cause 0-12345-67-8/90/12 to appear in the copyright line.
%
% ---------------------------------------------------------------------------------------------------------------
% This .tex source is an example which *does* use
% the .bib file (from which the .bbl file % is produced).
% REMEMBER HOWEVER: After having produced the .bbl file,
% and prior to final submission, you *NEED* to 'insert'
% your .bbl file into your source .tex file so as to provide
% ONE 'self-contained' source file.
%
% ================= IF YOU HAVE QUESTIONS =======================
% Questions regarding the SIGS styles, SIGS policies and
% procedures, Conferences etc. should be sent to
% Adrienne Griscti (griscti@acm.org)
%
% Technical questions _only_ to
% Gerald Murray (murray@hq.acm.org)
% ===============================================================
%
% For tracking purposes - this is V2.0 - May 2012

\documentclass{sig-alternate-05-2015}
\usepackage{epstopdf}

\begin{document}

% Copyright
\setcopyright{acmcopyright}
%\setcopyright{acmlicensed}
%\setcopyright{rightsretained}
%\setcopyright{usgov}
%\setcopyright{usgovmixed}
%\setcopyright{cagov}
%\setcopyright{cagovmixed}


\title{Classification of Optional Pratical Training (OPT) comments using a Naive bayes classifier}

%
% You need the command \numberofauthors to handle the 'placement
% and alignment' of the authors beneath the title.
%
% For aesthetic reasons, we recommend 'three authors at a time'
% i.e. three 'name/affiliation blocks' be placed beneath the title.
%
% NOTE: You are NOT restricted in how many 'rows' of
% "name/affiliations" may appear. We just ask that you restrict
% the number of 'columns' to three.
%
% Because of the available 'opening page real-estate'
% we ask you to refrain from putting more than six authors
% (two rows with three columns) beneath the article title.
% More than six makes the first-page appear very cluttered indeed.
%
% Use the \alignauthor commands to handle the names
% and affiliations for an 'aesthetic maximum' of six authors.
% Add names, affiliations, addresses for
% the seventh etc. author(s) as the argument for the
% \additionalauthors command.
% These 'additional authors' will be output/set for you
% without further effort on your part as the last section in
% the body of your article BEFORE References or any Appendices.

\numberofauthors{4} %  in this sample file, there are a *total*
% of EIGHT authors. SIX appear on the 'first-page' (for formatting
% reasons) and the remaining two appear in the \additionalauthors section.
%
\author{
% You can go ahead and credit any number of authors here,
% e.g. one 'row of three' or two rows (consisting of one row of three
% and a second row of one, two or three).
%
% The command \alignauthor (no curly braces needed) should
% precede each author name, affiliation/snail-mail address and
% e-mail address. Additionally, tag each line of
% affiliation/address with \affaddr, and tag the
% e-mail address with \email.
%
% 1st. author
\alignauthor 
Anand\\ 
       \email{a3anand@ucsd.edu}
% 2nd. author
\alignauthor Sampath\\
       \email{svelaga@ucsd.edu}
% 3rd. author
\alignauthor 
Jorge Garza\\
       \email{jgarzagu@ucsd.edu}
\and  % use '\and' if you need 'another row' of author names
% 4th. author
\alignauthor 
Adithya\\
       \email{akaravad@ucsd.edu}
}

\maketitle
\begin{abstract}
This paper aims to classify the optional practical training comments using a naive bayes classifier. We demonstrate the effectiveness of the Naive bayes approach and further enhance its performance using a kind of expectation\
maximisation algorithm. We explore how sentiments change over time, and also provide preliminary results that help in understanding how sentiments vary with ethnicity

\end{abstract}

\section{Introduction}


\subsection{Data collection}

\subsection{Data visualisation/Exploratory analysis}

\subsection{Dataset Labeling}
Since the original dataset is unlabeled, we manually labeled the first 900 comments as "support" or "oppose". Out of these, the first 600 were used for training, comments from 601-700 constituted the validation set and 700-900 were used for testing .\ 
We used validation set to pick the best possible model from amongst a pool of possible models. 
\subsection{Predictive task}
Our main goal here was to classify whether a given comment is  supporting or an opposing. In addition, based on the classifier we obtained, we also examined how the sentiment varied with time. Finally, we tried to examine the trends on an ethnicity basis. The main idea of this was to check if the pattern folows the general belief that most Americans oppose OPT extension, while people from other ethnicities support it.

\section{Related Literature}

\section{Algorithms and models tried for classification}
Broadly speaking, there are two classes of algorithms that could be tried to classify the comment labels - Supervised and unsupervised. For supervised learning, we tried hierarchical clustering and K means clustering. We used tf-idf frequencies, after eliminating the stop words, and the words that occured only once in the whole corpus. The results that were obtained were terrible, and we realised that clustering with simeple distance measure such as the Eucledian distance metric is very inappropriate for text classification.

Next, we implemented a naive bayes classifier using both unigram and bigram words. First, we pre-processed the words by removing the most common stop words and eliminated all punctuations. We then converted all uppercase letters to lowecase letters. 


\section{Results and discusision}

\section{Conclusions}
This paragraph will end the body of this sample document.
Remember that you might still have Acknowledgments or
Appendices; brief samples of these
follow.  There is still the Bibliography to deal with; and
we will make a disclaimer about that here: with the exception
of the reference to the \LaTeX\ book, the citations in
this paper are to articles which have nothing to
do with the present subject and are used as
examples only.
%\end{document}  % This is where a 'short' article might terminate

\section{Acknowledgments}
This section is optional; it is a location for you
to acknowledge grants, funding, editing assistance and
what have you.  In the present case, for example, the
authors would like to thank Gerald Murray of ACM for
his help in codifying this \textit{Author's Guide}
and the \textbf{.cls} and \textbf{.tex} files that it describes.


\section{References}
Generated by bibtex from your ~.bib file.  Run latex,
then bibtex, then latex twice (to resolve references)
to create the ~.bbl file.  Insert that ~.bbl file into
the .tex source file and comment out
the command \texttt{{\char'134}thebibliography}.
% This next section command marks the start of
% Appendix B, and does not continue the present hierarchy
\section{More Help for the Hardy}
% That's all folks!
\end{document}
